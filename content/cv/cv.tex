%-------------------------
% Resume in Latex
% Author : Nikolas Sibaev
% Using off of: https://github.com/sb2nov/resume
% License : MIT
%------------------------

\documentclass[letterpaper,11pt]{article}

\usepackage{latexsym}
\usepackage[empty]{fullpage}
\usepackage{titlesec}
\usepackage{marvosym}
\usepackage[usenames,dvipsnames]{color}
\usepackage{verbatim}
\usepackage{enumitem}
\usepackage[hidelinks]{hyperref}
\usepackage{fancyhdr}
\usepackage[english]{babel}
\usepackage{tabularx}
\input{glyphtounicode}


%----------FONT OPTIONS----------
% sans-serif
% \usepackage[sfdefault]{FiraSans}
% \usepackage[sfdefault]{roboto}
% \usepackage[sfdefault]{noto-sans}
% \usepackage[default]{sourcesanspro}

% serif
%\usepackage{CormorantGaramond}
\usepackage{charter}


\pagestyle{fancy}
\fancyhf{} % clear all header and footer fields
\fancyfoot{}
\renewcommand{\headrulewidth}{0pt}
\renewcommand{\footrulewidth}{0pt}

% Adjust margins
\addtolength{\oddsidemargin}{-0.5in}
\addtolength{\evensidemargin}{-0.5in}
\addtolength{\textwidth}{1in}
\addtolength{\topmargin}{-.5in}
\addtolength{\textheight}{1.0in}

\urlstyle{same}

\raggedbottom
\raggedright
\setlength{\tabcolsep}{0in}

% Sections formatting
\titleformat{\section}{
  \vspace{-4pt}\scshape\raggedright\large
}{}{0em}{}[\color{black}\titlerule \vspace{-5pt}]

% Ensure that generate pdf is machine readable/ATS parsable
\pdfgentounicode=1

%-------------------------
% Custom commands
\newcommand{\resumeItem}[1]{
  \item\small{
    {#1 \vspace{-2pt}}
  }
}

\newcommand{\resumeSubheading}[4]{
  \vspace{-2pt}\item
    \begin{tabular*}{0.97\textwidth}[t]{l@{\extracolsep{\fill}}r}
      \textbf{#1} & #2 \\
      \textit{\small#3} & \textit{\small #4} \\
    \end{tabular*}\vspace{-7pt}
}

\newcommand{\resumeSubSubheading}[2]{
    \item
    \begin{tabular*}{0.97\textwidth}{l@{\extracolsep{\fill}}r}
      \textit{\small#1} & \textit{\small #2} \\
    \end{tabular*}\vspace{-7pt}
}

\newcommand{\resumeProjectHeading}[2]{
    \item
    \begin{tabular*}{0.97\textwidth}{l@{\extracolsep{\fill}}r}
      \small#1 & #2 \\
    \end{tabular*}\vspace{-7pt}
}

\newcommand{\resumeSubItem}[1]{\resumeItem{#1}\vspace{-4pt}}

\renewcommand\labelitemii{$\vcenter{\hbox{\tiny$\bullet$}}$}

\newcommand{\resumeSubHeadingListStart}{\begin{itemize}[leftmargin=0.15in, label={}]}
\newcommand{\resumeSubHeadingListEnd}{\end{itemize}}
\newcommand{\resumeItemListStart}{\begin{itemize}}
\newcommand{\resumeItemListEnd}{\end{itemize}\vspace{-5pt}}


%-------------------------
%Toggle Languages & Length
\usepackage{etoolbox}
\newtoggle{ENGLISH}
\newtoggle{LONGCV}
%\toggletrue{ENGLISH}
%\togglefalse{ENGLISH}
%\toggletrue{LONGCV}

%-------------------------

%-------------------------------------------
%%%%%%  RESUME STARTS HERE  %%%%%%%%%%%%%%%%%%%%%%%%%%%%


\begin{document}

%----------HEADING----------
% \begin{tabular*}{\textwidth}{l@{\extracolsep{\fill}}r}
%   \textbf{\href{http://sourabhbajaj.com/}{\Large Sourabh Bajaj}} & Email : \href{mailto:sourabh@sourabhbajaj.com}{sourabh@sourabhbajaj.com}\\
%   \href{http://sourabhbajaj.com/}{http://www.sourabhbajaj.com} & Mobile : +1-123-456-7890 \\
% \end{tabular*}

\begin{center}
    \textbf{\Large \scshape Nikolas Sibaev} \\ \vspace{1pt}
    \small +49 176 3157 5632 $|$
    \href{https://sibaev.de}{\underline{sibaev.de}} $|$
    \href{mailto:nikolas@sibaev.de}{\underline{nikolas@sibaev.de}} $|$
    \href{https://linkedin.com/in/nikolas-sibaev}{\underline{linkedin.com/in/nikolas-sibaev}} $|$
    \href{https://github.com/3213-cmd/}{\underline{github.com/3213-cmd/}}
\end{center}


%-----------EDUCATION-----------
%START
\iftoggle{ENGLISH}
{
\section{Education}
\resumeSubHeadingListStart
}
{
\section{Ausbildung}
\resumeSubHeadingListStart
}
\iftoggle{ENGLISH}
{
  \resumeSubheading
  {Ludwig Maximilian University of Munich}{Munich, Germany}
  {Bachelor of Science in Computer Science, Minor in Philosophy}{April 2017 -- December 2022}
    %\resumeSubheading
     % {Blinn College}{Bryan, TX}
      %{Associate's in Liberal Arts}{Aug. 2014 -- May 2018}
}
{
  \resumeSubheading
  {Ludwig-Maximilians-Universität-München}{München, Deutschland}
  {Bachelor of Science in Informatik mit Nebenfach Philosophie}{April 2017 -- Dezember 2022}
}

%END
\resumeSubHeadingListEnd
%-----------EXPERIENCE-----------
%START
\iftoggle{ENGLISH}
{
\section{Experience}
\resumeSubHeadingListStart
}
{
\section{Erfahrung}
\resumeSubHeadingListStart
}
\iftoggle{ENGLISH}
  {
    \resumeSubheading
      {Junior Cloud DevOps Engineer}{October 2023 -- Present}
      {Eviden}{Munich, Germany}
      \resumeItemListStart
        \resumeItem{Developed a on Microsoft Azure hosted WebApp using the C\# Blazor Framework}
        \resumeItem{Assisting a customer with the building of a CI/CD pipeline using GitHub Actions and Azure DevOps}
        \resumeItem{Wrote Bash and Python scripts for different one-off tasks}
        \resumeItem{Learning the foundations of various Cloud and DevOps related technologies}
        \resumeItemListEnd
      }
      {
        \resumeSubheading
      {Junior Cloud DevOps Engineer}{Oktober 2023 -- Heute}
      {Eviden}{München, Deutschland}
      \resumeItemListStart
        \resumeItem{Entwickeln einer auf Microsoft Azure gehosteten WebApp mittels des C\# Blazor Frameworks}
        \resumeItem{Unterstützung eines Kunden beim Aufbau einer CI/CD Pipeline mittels GitHUb Actions und Azure DevOps}
        \resumeItem{Schreiben von Bash und Python Skripten für diverse einmalige Aufgaben}
        \resumeItem{Erlernen von Grundlagen in unterschiedlichen Cloud und DevOps bezogenen Technologien}
        \resumeItemListEnd
      }
% -----------Multiple Positions Heading-----------
%    \resumeSubSubheading
%     {Software Engineer I}{Oct 2014 - Sep 2016}
%     \resumeItemListStart
%        \resumeItem{Apache Beam}
%          {Apache Beam is a unified model for defining both batch and streaming data-parallel processing pipelines}
%     \resumeItemListEnd
%    \resumeSubHeadingListEnd
%-------------------------------------------

    \iftoggle{ENGLISH}
      {
    \resumeSubheading
      {Intern Product Management}{August 2020 -- March 2022}
      {Amazon Fresh}{Munich, Germany}
      \resumeItemListStart
        \resumeItem{Supported set-up of new pricing mechanisms}
        \resumeItem{Prototyped projects for automated data processing}
        \resumeItem{Owned weekly SQL and data analytics office hours}
        \resumeItem{Ran demo sessions with international senior stakeholders}
        \resumeItem{Development of internationaly used reports on customer insights}
    \resumeItemListEnd
  }
  {

    \resumeSubheading
      {Werkstudent Produkt Management}{August 2020 -- März 2022}
      {Amazon Fresh}{München, Deutschland}
      \resumeItemListStart
        \resumeItem{Unerstützung beim Aufsetzen neuer Pricing Mechanismen}
        \resumeItem{Entwicklung mehrer Prototypen zur automatischen Datenverarbeitung}
        \resumeItem{Halten einer wöchentlichen SQL und Data-Analytics Sprechstunde}
        \resumeItem{Halten von Demo-Sitzungen mit internationalen Senior Stakeholdern}
        \resumeItem{Entwickeln international genutzter Berichte über die Kunden Einblicke}
    \resumeItemListEnd
  }

\iftoggle{ENGLISH}
      {
    \resumeSubheading
      {Technical Support}{April 2020 -- July 2020}
      {Bayrische TelemedAllianz}{Baar-Ebenhausen, Germany}
      \resumeItemListStart
        \resumeItem{Technical support for the telecommunications platform \emph{Doccura}}
        \resumeItem{Testing and planning of new features}
        \resumeItem{Integrating the ticketing platform \emph{Freshdesk} as well as traing employees on its usage}
        \resumeItem{Creating the documentation and FAQ}
      \resumeItemListEnd
}
{
    \resumeSubheading
      {Technischer Support}{April 2020 -- July 2020}
      {Bayrische TelemedAllianz}{Baar-Ebenhausen, Deutschland}
      \resumeItemListStart
        \resumeItem{Technischer Support für die Telekommunikationsplattform \emph{Doccura}}
        \resumeItem{Testen und mitbestimmen von neuen Features}
        \resumeItem{Integrierung der Ticketingplattform \emph{Freshdesk} sowie Einarbeitung der Mitarbeiter}
        \resumeItem{Erstellen von Dokumentation und FAQ}
      \resumeItemListEnd

    }
%LONGCV START
\iftoggle{LONGCV}{
\iftoggle{ENGLISH}
      {
    \resumeSubheading
      {Intern Marketing}{September 2018 -- March 2020}
      {Verlag C.H. Beck oHG}{Munich, Germany}
      \resumeItemListStart
        \resumeItem{Creation of advertisments using Adobe Photoshop and InDesign}
        \resumeItem{Proofreading and correctional work}
        \resumeItem{Maintenance of online content}
        \resumeItem{Automation of manual advertisment creation steps}
      \resumeItemListEnd
}
{
    \resumeSubheading
      {Werkstudent Marketing}{September 2018 -- März 2020}
      {Verlag C.H. Beck oHG}{München, Deutschland}
      \resumeItemListStart
        \resumeItem{Erstellen von Werbeanzeigen mittels Adobe Photoshop und InDesign}
        \resumeItem{Korrekturlesen und Berichtigungsarbeit}
        \resumeItem{Wartung von Online Content}
        \resumeItem{Automatisierung manueller Werbeanfertigungsschritte}
      \resumeItemListEnd

    }

%LONCV END
  }

%END
\resumeSubHeadingListEnd
%-----------PROJECTS-----------

\iftoggle{ENGLISH}
{
\section{Projects}
    \resumeSubHeadingListStart
}
{
\section{Projekte}
    \resumeSubHeadingListStart
}
\iftoggle{ENGLISH}
{
\resumeProjectHeading
          {\textbf{Mailman} $|$ \emph{Clojure, Clojurescript, GraphQL, PostgreSQL, Docker}}{July 2023 -- Present}
          \resumeItemListStart
            \resumeItem{Private ongoing project with the goal to create a full-stack app for faciliating movement of email providers}
            \resumeItem{Frontend is written in Clojurescript and uses Material-UI components with Reagent}
            \resumeItem{Backend is written in Clojure and utilizies several libraries such as NextJDBC, Honeysql and Lacinia}
            \resumeItem{Features GraphQL API}
            \resumeItem{Goal is to have a project which can be easily deployed locally using Docker compose}
            \resumeItem{Project is used as an exercise to build practical experience with Clojure}
          \resumeItemListEnd
}
{
\resumeProjectHeading
          {\textbf{Mailman} $|$ \emph{Clojure, Clojurescript, GraphQL, PostgreSQL, Docker}}{Juli 2023 -- Heute}
          \resumeItemListStart
            \resumeItem{Privates Projekt mit dem Ziel eine Fullstack App bereitzustellen, die den Wechsel von Mail Providern erleichert}
            \resumeItem{Frontend ist in Clojurescript geschrieben und nutzt Reagent zusammen mit Material-Ui Komponenten}
            \resumeItem{Backend ist in Clojure geschrieben und nutzt divesrse Bibliotheken wie z.B: NextJDBC, HoneySQL und Lacinia}
            \resumeItem{Beinhaltet eine GraphQL API}
            \resumeItem{Projekt soll lokal einfach mittels Docker compose zum Laufen gebracht werden}
            \resumeItem{Projekt dient primär dazu praktische Erfahrung in Clojure aufzubauen}
          \resumeItemListEnd
}
\iftoggle{ENGLISH}
{
\resumeProjectHeading
          {\textbf{Certmand} $|$ \emph{Python, VMware, Linux, PKI, SQLite, SQLAlchemy, REST-API, Swagger-Ui}}{Oktober 2021 -- Dezember 2022}
          \resumeItemListStart
            \resumeItem{Worked in a group of three as part of our bachelor thesis}
            \resumeItem{Developed a an automatic certificate management system for VMware based environments}
            \resumeItem{Service that automatically renewed expiring certificates}
            \resumeItem{Used a plugin based system to support different kinds of components}
            \resumeItem{Environment was build from ground up including the network components such as the DNS or Proxy}
          \resumeItemListEnd
}
{
\resumeProjectHeading
          {\textbf{Certmand} $|$ \emph{Python, VMware, Linux, PKI, SQLite, SQLAlchemy, REST-API, Swagger-Ui}}{Oktober 2021 -- Dezember 2022}
          \resumeItemListStart
            \resumeItem{Gruppenarbeit mit zwei Kommilitionen im Rahmen unserer Bachelorarbeiten}
            \resumeItem{Entwickeln eines automatischen Zertifikatsmanagementsystems für VMware basierte Umgebungen}
            \resumeItem{Service der automatisch ablaufende Zertifikate erneuert}
            \resumeItem{Plugin Architektur um unterschiedliche Arten von Komponenten zu unerstützen}
            \resumeItem{Umgebung war von grundauf selbst aufgebaut einschließlich der Netzwerkkomponenten wie DNS und Proxy}
          \resumeItemListEnd
        }
%LONGCV START
        \iftoggle{LONGCV}
        {
          \iftoggle{ENGLISH}
          {
\resumeProjectHeading
          {\textbf{Bashni Client} $|$ \emph{C, TCP, IP, ClientServer, Pipes, Sockets, Makefile}}{October 2020 -- February 2021}
          \resumeItemListStart
            \resumeItem{Group project as part of a practical course on operating systems during my studies}
            \resumeItem{Developed a client for the board game Bashni which communicated with a provided server}
            \resumeItem{Learned to use system-level programming concepts}
            \resumeItem{Included custom AI that was later trialed against the AIs of other student groups}
          \resumeItemListEnd
          }
          {
\resumeProjectHeading
          {\textbf{Bashni Client} $|$ \emph{C, TCP, IP, ClientServer, Pipes, Sockets, Makefile}}{Oktober 2020 -- Februar 2021}
          \resumeItemListStart
            \resumeItem{Gruppenprojekt in Rahmen des Systempraktikums während des Studiums}
            \resumeItem{Entwickeln eines Clients für das Brettspiel Bashni}
            \resumeItem{Erlernen von Konzepten der systemnahen Programmierung}
            \resumeItem{Beeinhaltete eigensentwicklete KI welche späteter gegen die KIs anderer Studentengruppen antrat}
          \resumeItemListEnd
          }


%LONGCVEND
}



%END
\resumeSubHeadingListEnd
%-----------PROGRAMMING SKILLS-----------
\iftoggle{ENGLISH}
{
\section{Technical Skills}
 \begin{itemize}[leftmargin=0.15in, label={}]
    \small{\item{
     \textbf{Languages}{: Python, SQL, Clojure, Clojurescript, C, Java, HTML/CSS} \\
     \textbf{Frameworks}{: Blazor, SQLAlchemy, Reagent, Tornado, Material-UI, FastAPI} \\
     \textbf{Developer Tools}{: Git, Docker, Azure, AWS, EMACS, PyCharm, GitHub Actions, Terraform, Postman} \\
     \textbf{Libraries}{: SQLAlchemy, Magit, Lacinia, HoneySQL}
    }}
 \end{itemize}
}
{
\section{Technische Fähigkeiten}
 \begin{itemize}[leftmargin=0.15in, label={}]
    \small{\item{
     \textbf{Sprachen}{: Python, SQL, Clojure, Clojurescript, C, Java, HTML/CSS} \\
     \textbf{Frameworks}{: Blazor, SQLAlchemy, Reagent, Tornado, Material-UI, FastAPI} \\
     \textbf{Entwickler Werkzeuge}{: Git, Docker, Azure, AWS, EMACS, PyCharm, GitHub Actions, Terraform, Postman} \\
     \textbf{Bibliotheken}{: SQLAlchemy, Magit, Lacinia, HoneySQL}
    }}
 \end{itemize}
}


%-------------------------------------------
\end{document}
